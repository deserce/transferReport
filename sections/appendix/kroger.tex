\appendix
\addcontentsline{toc}{chapter}{Appendix}
\chapter{Defects}
In practice, thermal vibrations lead to ionic solids deviating from their perfect crystal structures. 
The mobility and concentration of these defects is often strongly linked to properties of interest.

Defects can be subcategorised by their dimensionality as listed in Table A.1:

\begin{table}[bh!]
\centering
\label{tab:defecthierarchy}
\caption{The defect hierarchy \citep{Carter2013}}
\begin{tabular}{@{}lll@{}}
\toprule
Type            & Dimensionality  & Example\\
\midrule
Point defect    & Zero            & Vacancy\\
Line defect     & One             & Dislocation\\
Planar defect   & Two             & Grain boundary\\
Volume defect   & Three           & Pore\\
\bottomrule
\end{tabular}
\end{table}

It is only point defects which are considered within this report.
\newpage
\section{Point defects}
The formation energy of a number of point defect types have been calculated within this report.


\begin{labeling}{\textbf{Interstitial}}
\item[\textbf{Vacancy}] Sites not occupied by an ion, which would be in the perfect crystal structure. The movement of a neighbouring ion into a vacancy creates a new vacancy. As such, it is convenient to assign diffusivity values to defects, similar to the treatment of electron holes as mobile in metals. The energy associated with this defect is the energy required to move the ion to a point an infinite distance from the defect (i.e. the removed ion is no longer interacting with the crystal).

\item[\textbf{Interstitial}] The presence of an ion at a site not occupied as prescribed by the perfect crystal structure. These defects are typically high energy, especially with large ions, as the neighboring ions are displaced to accommodate the new ion.
\item[\textbf{Impurity}] The introduction of ions not found in the perfect crystal. This can either be at an interstitial site (interstitial impurity), or in lieu of an ion in the perfect crystal (substitutional defect).
For substitutional defects, they can effectively be considered to be combination of a vacancy at the lattice site, followed by an interstitial impurity at that same lattice site.
\end{labeling}

With the exception of substitutional impurities which may conserve charge, (e.g. \ce{Ni^{2+}} impurity at \ce{Fe^{2+}} site), the creation of point defects has an associated charge. The creation of an oxygen vacancy has a charge of +2 for example.

\newpage
\section{Kr\"oger-Vink Notation}
Kr\"oger-Vink Notation \citep{Kroger1958} is conventionally used to define how point defects form or interact with one another.
The notation uses the structure: \begin{equation}
\ch{S^c_{p}}
\label{eq:kroger}
\end{equation} with $S$ as the species, $c$ as the charge of that species, and $p$ as the position in which that species is found.
The species, $S$, indicates whether the species is chemical (denoted by the appropriate chemical symbol), or a vacancy (denoted \ch{V} or \ch{Va} to distinguish from Vanadium).
Null reactant is indicated by the symbol $\varnothing$.

Charge is described by the superscripts \ch{S^*} and \ch{S'} to represent positive and negative charges respectively. Repetition of the superscript indicates charges of larger magnitude.
\ch{S'}, \ch{S''}, and \ch{S'''} represent charges of 1-, 2- and 3- respectively as an example.
Neutral charges are expressed by \ch{S^x}.

Finally, the location of the defect is given as a subscript $p$.
This can either indicate the defect is present at a regular lattice site, in which case the chemical symbol of the species which occupies the site is given (\ch{H,~ He,~ Li,~ \ldots}), or indicate the defect is in an interstitial site, indicated by the subscript $i$.
\vfill
\begin{table}[h]
\centering
\caption{Summary of Kr\"oger-Vink notation, corresponding to Equation \ref{eq:kroger} \citep{Carter2013}}
\begin{tabular}{@{}lll@{}}
\toprule
Symbol          & Possible Values                & Notes\\
\midrule
S              & \ch{H,~ He,~ Li}, \ldots                                              & Indicates a chemical species\\
                & V                                                                             & Vacancy. If Vanadium is present, use Va for Vacancy.\\
                & $\varnothing$                                                                 & Null reactant   \\
c               & $\bullet,~ \bullet\!\bullet,~ \bullet\!\bullet \bullet$, \ldots        & Positively charged species: 1+, 2+, 3+, ...       \\
                & $\prime,~ \prime \prime,~ \prime \prime \prime$, \ldots               &  Negatively charged species: 1-, 2-, 3-, ...      \\
                & $\times$                                                                      &  Neutrally charged species.      \\
p               & \ch{H,~ He,~ Li}, \ldots                                               &  Defect at position occupied by species p in ideal crystal.\\
                & i                                                                             &  Defect in interstitial space.\\
\bottomrule
\end{tabular}
\label{tab:krogerSummary}
\end{table}

\newpage
\section{Associated defects for \ce{Li4Mn2O5}}
As crystals must be neutrally charged, defect formation will often occur in groups, with a number of fundamental defects occurring to form charge neutral defects. Here the possible defects for \ce{Li4Mn2O5} are listed.
In the case of mixed systems, defects have only been considered using partial occupancy approaches.
As such, the following equations still apply, but with pseudo-ions instead of \ce{Fe}.

\begin{labeling}{\textbf{Schottky-like}}
\item[\textbf{Full Schottky}] Several vacancies are formed corresponding to the stoichiometric ratios of the perfect crystal, yielding a formula unit of the lattice.
 In \ce{Li4Mn2O5}, a full Schottky consist of four lithium vacancies, two manganese vacancies, five oxygen vacancies, and the formation of a molecule of \ce{Li4Mn2O5} at the surface of the bulk crystal.
\begin{equation}
\varnothing \rightarrow \ce{4 V^\prime_{Li} +  3 V^{\prime \prime \prime}_{Mn} + 4V^{\bullet \bullet}_{O} + Li4Mn2O5}
\end{equation}
\item[\textbf{Schottky-like}] As with Schottky defects, but the ratio of defects (and therefore the molecule formed) has a different stoichiometry to the bulk crystal.
For \ce{Li4Mn2O5}, there are two Schottky-like defects:
\begin{align}
\varnothing &\rightarrow \ce{2V^\prime_{Li} + V^{\bullet \bullet}_{O} + Li2O} \\
\varnothing &\rightarrow \ce{2V^{\prime \prime \prime}_{Mn} + 3 V^{\bullet \bullet}_{O} + Mn2O3}
\end{align}
\item[\textbf{Frenkel}] The formation of a vacancy, with stoichiometry and charge of the bulk material conserved by the formation of a corresponding interstitial defect of the same species:
\begin{align}
\varnothing &\rightarrow \ce{V^\prime_{Li} +  Li^{\bullet}_{i}}\\
\varnothing &\rightarrow \ce{V^{\prime \prime \prime}_{Mn} +  Mn^{\bullet\bullet\bullet}_{i}}\\
\varnothing &\rightarrow \ce{V^{\bullet\bullet}_{O} +  O^{\prime \prime}_{i}}
\end{align}
\item[\textbf{Antisite}] The exchange of two ions between their normal lattice sites. These are of significant interests for \ce{LiFePO4}, as \ce{Li/Fe} antisite defects can form in \ce{Li} channels and prevent \ce{Li} migration.
\begin{align}
\varnothing &\rightarrow \ce{Li^\prime_{Fe} +  Fe^{\bullet}_{Li}}
\end{align}
\end{labeling}
