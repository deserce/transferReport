\chapter{Introduction}
\section{Background: Energy storage}
The demand for ever improving energy storage solutions is driven by both commercial interests and urgent societal need.
In order to curtail \ce{CO2} emissions and limit the impact of global warming, a shift from fossil-fuel based energy generation towards renewable energy is vital.\cite{Goodenough2013}
Unfortunately, a characteristic of many renewable energy sources, such as wind and solar energy, is the intermittent nature of its generation.\cite{Goodenough2010a}
As such, it is necessary to decouple the generation and consumption of the energy produced by these means via intermediate storage.

Vehicle electrification may displace conventional combustion engine driven transport, another major source of greenhouse gases and pollution.
While some commercial success has being enjoyed for electric vehicles in recent years, safety concerns and insufficient energy density to overcome the ``range anxiety'' of most consumers demands further development of battery technologies for widespread adoption of electric vehicles to occur.

Finally, consumer demand for increasingly powerful, thin, and light portable electronics with larger times between charges has pushed current battery chemistries to their limits.
With the majority of developments in batteries for portable electronics in recent years arising not from improvements in their chemistries, but by improved manufacturing processes with diminishing returns, the need for a fundamentally different battery chemistry may arise in the near future.
Indeed, in recent years a number of manufacturers have, in a drive to yield more performance from the same fundamental chemistries, produced cells which ultimately proved to be unsafe.\cite{Loveridge2018}

Each of the above use cases will prioritise different properties in their energy storage solutions. 
While safety and energy density are paramount in portable electronics and electric vehicles, low cost is the primary driver in load-levelling applications.


\newpage
\section{Li-ion Batteries}
\renewcommand{\thefootnote}{\fnsymbol{footnote}}
Lithium, being the lightest and most electropositive metal, is well suited to applications in high density energy storage.
The first commercial Li-ion battery was developed in the 1980's by John Goodenough\cite{Mizushima1981} and manufactured by Sony Co. in 1991.\cite{Li2018}
This development, coupled with the rapid decrease in the size of transistors, sparked the portable electronics revolution.

In the years since, a range of battery chemistries have been developed, tailored to specific applications.

\subsection{Anatomy of an electrochemical cell}
Figure \ref{fig:Goodenough} illustrates a typical electrochemical cell, speciifically the first commercial Li-ion battery.
The cell consists of an anode and cathode\footnote{It is conventional when discussing battery materials to refer to the electrode which is positive during discharge as the anode, and the electrode which is negative as the cathode, regardless of charge state.},
which are electrically isolated from one another but able to freely exchange ions via an electrolyte.
The cell is charged by applying an external potential across the cell, forcing electrons to the anode.
\ce{Li+} ions are extracted from the \ce{LiCOO2} cathode, diffuse between the electrodes via an electrolyte (\ce{LiFP6 in organic solvent} typically), and intercalate into the anode structure to maintain charge neutrality.

During discharge, the reverse process occurs, with \ce{Li+} ions driven by a difference in electrical potential between the electrodes, and electrons travelling via an external circuit, allowing useful work to be done.
When discharging, the \ce{Li+} ions are driven back to the cathode by the difference in chemical potential between the electrodes.
Electrons move back to the cathode via an external circuit, allowing work to be done.
Figure \ref{fig:Goodenough} illustrates the first commercial Li-ion battery, consisting of a \ce{LiCoO2} cathode, a \ce{LiFP6} electrolyte, and a graphite anode.


\begin{figure}
\centering
\begin{subfigure}{\linewidth}
  \centering
  \includegraphics[width=0.7\linewidth, trim=1cm 1cm 1cm 1cm, clip]{figures/batteryCharge/batteryCharge}
  \caption{Charging}
  \label{fig:GoodenoughCharging}
\end{subfigure}

\begin{subfigure}{\linewidth}
  \centering
  \includegraphics[width=0.7\linewidth, trim=1cm 1cm 1cm 1cm, clip]{figures/batteryDischarge/batteryDischarge}
  \caption{Discharging}
  \label{fig:GoodenoughDischarging}
\end{subfigure}
\caption[Li-ion battery schematic]{A schematic of a common Li-ion battery. During charging (a), electrons are driven to the anode by an external potential. \ce{Li+} ions move from the cathode to the anode to charge balance. During discharge (b), \ce{Li+} ions are driven to the cathode by a difference in chemical potential. The electrodes are electrically isolated by the separator, and are forced to instead travel via an external circuit and provide work.} 
\label{fig:Goodenough}
\end{figure}

\newpage
\section{Electrolyte materials}
Explaination of electrolytes.
\subsection{Liquid electrolytes}
Currently widely used.
Significant concerns pertaining to safety, particularly for EV's.
Narrow stable voltage windows may be a limiting factor for newer cathodes.
\subsection{Solid electrolytes}
\citet{Bachman2016,Manthiram2017a,Janek2016,Famprikis2019,Zhang2018}
\section{Anode materials}
Li theoretically ideal.

Graphite commonly used

Metal oxides

Silicon

\section{Cathode materials}
\subsection{LiMO2}
LiCoO2

LiNiO2

LiMnO2

NMC

\subsection{LiFePO4}

\subsection{Spinel}

\subsection{Alternatives}
Na/Mg-ion

Li-air

\subsection{Limitations}

\section{Li-rich cathodes}

Motivation, definition.

Anion redox. \cite{Yahia2019}


\subsection{Li2MnO3}

\subsection{Li-rich NMC}

\subsection{Li4Mn2O5}
\subsection{Li2MnO2F}




