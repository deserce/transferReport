\chapter{Defect chemistry and \ce{Li} ion diffusion in \ce{Li4Mn2O5}}
\section{Background}
\newpage
\begin{figure}
\centering
\includegraphics[width=0.7\linewidth]{figures/structures/Li4Mn2O5}
\caption[Crystal structure of \ce{Li4Mn2O5}]{Crystal structure of \ce{Li4Mn2O5}. Li ions: green; Mn ions: purple; \ce{O} ions: red.}
\end{figure}
\section{Structure generation and modelling}
\ce{Li4Mn2O5}, shown in Figure, has an average \ce{MnO} rock-salt structure with the random substitution of 2/\nth{3}\textsuperscript{s} of \ce{Mn} sites with Li, charge compensated by 1/\nth{6} of oxygen sites being vacant.
While no ordering is noted in experimental work,\cite{Freire2016,Diaz-Lopez2018a} computational studies have identified a number of ordered structures which lower energies than pseudo-random structures generated for comparison.\cite{Diaz-Lopez2017,Bhandari2019}
It is worth noting that these studies, which utilised \textit{ab initio} methods, had no prerequisite for potential models robust in a wide range of local environments as is the case for such systems to be studied in this work.

\newpage

\section{Tables}

\begin{table}[t]
\centering
\caption[Two-body short-range potential parameters for \ce{Li4Mn2O5}]{Two-body short-range potential parameters for \ce{Li4Mn2O5}.\cite{Ammundsen1999}}
\begin{tabular}{@{}lS[table-format=5.2]S[table-format=1.4]S[table-format=2.1]S[table-format=1.2]S[table-format=5.1]@{}}
\toprule
\textbf{Interaction} &\multicolumn{1}{c}{\textbf{A} (\si{\electronvolt})}   & \multicolumn{1}{c}{$\boldsymbol{\rho}$ (\AA)} & \multicolumn{1}{c}{\textbf{C} (\si{\electronvolt \angstrom \tothe{6}})} & \multicolumn{1}{c}{\textbf{Y} (e)}& \multicolumn{1}{c}{\textbf{K} (\si{\electronvolt \angstrom \tothe{-2}})}\\
\midrule
\ce{Li+ -\;O^2-}   & 426.48  & 0.300                          & 0.0  & 1.0        & 99999.0 \\
\ce{Mn^3+ -\;O^2-}       & 1267.5  & 0.3240                         & 0.0  & 3.00       & 850.0   \\
\ce{O^2- -\;O^2-}        & 22764.3 & 0.1490                         & 43.0 & -2.96      & 57.0\\
\bottomrule
\end{tabular}
\end{table}

\newcommand{\tableline}{\multicolumn{1}{c}{--}}
\newcommand{\mc}[1]{\multicolumn{1}{c}{#1}}

\begin{table}[t]
\centering
\caption{Calculated lattice parameters of \ce{Li4Mn2O5} using mean-field approximation.}
\begin{tabular}{lS[table-format=1.4]S[table-format=1.4]S[table-format=1.4]S[table-format=2.1]S[table-format=2.1]S[table-format=2.1]}
\toprule
\textbf{Lattice parameter} &\mc{\textbf{a} (\AA)}   & \mc{\textbf{b} (\AA)} & \mc{\textbf{c} (\AA)}& \mc{$\boldsymbol{\alpha}$ (\si{\degree})} & \mc{$\boldsymbol{\beta}$ (\si{\degree})} & \mc{$\boldsymbol{\gamma}$ (\si{\degree})}\\
\midrule
Experimental \cite{Freire2016} & 4.1733  & \tableline & \tableline & 90.0 & \tableline & \tableline \\
Mean-field                     & 4.0705  & \tableline & \tableline & 90.0 & \tableline & \tableline \\ 
\textit{\% $\Delta$}                 & -2.46   & \tableline & \tableline & \tableline & \tableline & \tableline \\\bottomrule
\end{tabular}
\end{table}

\section{Chapter summary}