\chapter*{Abstract}
The requirement for next-generation battery materials is driven by the move towards electric vehicles and the intermittent nature of renewable energy sources.
With current technologies at their technical limits, and a strong need to decouple energy storage from the availability of limited element supplies, the development of novel battery materials is urgently required.
An understanding of the fundamental mechanisms and phenomena which underpin the performance of these materials can identify new candidates and aid in the improvement and commercialisation of known materials.
Computational techniques can offer atomic-scale insights into defect formation and ion mobility in battery materials, providing an understanding that is difficult to extract by experimental studies alone.
In this project, energy minimisation and molecular dynamics techniques have been applied to a disordered rock-salt, \ce{Li4Mn2O5}, in order to characterise its performance as a cathode material with high energy density.
The disordered nature of this system raise a number of issues which are challenging for the use of conventional computational methods, highlighting the requirement for new structure generation techniques which account for the energetics of local environments.
The study of an ordered analogue of \ce{Li4Mn2O5} provided insight into the energetics of defect formation, suggesting low concentrations of Schottky and Frenkel defects.
The ordered nature of the structure gave rise to a difference in lithium vacancy formation energies, which in turn led to highly anisotropic migration pathways.
